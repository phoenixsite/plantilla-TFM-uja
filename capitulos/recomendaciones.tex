\chapter[Consideraciones elaboración TFG]{Consideraciones generales para la elaboración de un Trabajo Fin de Título}

\section{Normativa de la Universidad de Jaén y la Escuela Politécnica Superior de Jaén}

El \textsc{tft} lo rigen dos normativas: 
\begin{itemize}
  \item una a nivel general de la UJA (\href{https://www.ujaen.es/gobierno/secgen/sites/gobierno_secgen/files/uploads/CG25_ANEXO04_Normativa_Trabajos\%20_TFG_TFM_otros_TFT_UJA\%20CG\%2025\%20de\%205\%20junio\%202017.pdf}{Normativa marco UJA aprobada en Consejo de Gobierno (2017)}\footnote{\url{https://www.ujaen.es/gobierno/secgen/sites/gobierno_secgen/files/uploads/CG25_ANEXO04_Normativa_Trabajos\%20_TFG_TFM_otros_TFT_UJA\%20CG\%2025\%20de\%205\%20junio\%202017.pdf}}) y 
  \item otra complementaria a nivel de la Escuela Politécnica Superior de Jaén  (\href{https://eps.ujaen.es/sites/centro_epsj/files/uploads/documents/normativa/Normativa_TFG_TFM_EPSJ_aprobada\%20Junta\%20de\%20Escuela\%2013\%20sept\%202017.pdf}{Normativa EPSJ aprobada en Junta de Escuela (2017)}\footnote{\url{https://eps.ujaen.es/sites/centro_epsj/files/uploads/documents/normativa/Normativa_TFG_TFM_EPSJ_aprobada\%20Junta\%20de\%20Escuela\%2013\%20sept\%202017.pdf}}). 
\end{itemize}

Toda la información anterior puede encontrarse en la \href{https://eps.ujaen.es/principal/trabajo-fin-de-grado-master}{web de la Escuela Politécnia Superior de Jaén}\footnote{\url{https://eps.ujaen.es/principal/trabajo-fin-de-grado-master}}

\section{Normas de estilo}
Las normas de estilo para la realización del TFT se adaptarán en función de la modalidad de
TFT elegida.

\begin{itemize}
\item Los Proyectos de Ingeniería deberán presentar una estructura y contenido adaptados
a la norma UNE 157001 “Criterios generales para la elaboración de proyectos” u otra
derivada de ésta, en función de la naturaleza del proyecto.
Estos aspectos podrán detallarse bien en la guía docente y en la propuesta del TFG que
presente el departamento y apruebe la Comisión de TFG de la EPSJ. Por ejemplo, para
el caso de los sistemas de información informatizados (sistemas informáticos, sistemas
de información geográfica, etc.) está la norma derivada de la anterior UNE 157801
“Criterios generales para la elaboración de proyectos de sistemas de información”).
En general, el TFG deberá incluir los siguientes documentos básicos: Índice General,
Memoria, Anexos, Planos, Pliego de Condiciones , Estado de Mediciones, Presupuesto
y, cuando proceda, Estudios con Entidad Propia, presentados en el orden indicado. Los
planos, en caso de ser necesarios (los de sistemas de información, p.e., pueden no
llevar planos) se adecuarán a las normas UNE de Dibujo Técnico.  
\item Los Estudios Técnicos, debido a su singularidad, podrán tener una estructura variable
según su naturaleza. Como mínimo, deberían constar de los apartados siguientes:
Introducción, Objetivos, Discusión, Conclusiones, Planos y Anexos (si proceden) y
Bibliografía.
\item Los Trabajos teóricos o experimentales se estructurarán, en la medida de lo posible,
en los siguientes apartados: Introducción, Antecedentes, Objetivos, Material y
Métodos, Resultados y Discusión, Conclusiones, Planos y Anexos (si proceden) y
Bibliografía.
\end{itemize}

\begin{enumerate}
\item Idioma. La memoria del TFG se podrá elaborar en un idioma distinto al castellano, bajo
  petición del estudiante y del tutor/a a la Comisión de TFG de la Escuela, siempre que el idioma
  elegido por el alumno se encuentre entre los que se han utilizado en la impartición del grado.
  En este caso, se deberá proporcionar al menos un resumen con la introducción y las
  conclusiones del TFG en castellano. La Comisión de TFG de la Escuela valorará las peticiones, y
  accederá a ellas siempre que se tenga la posibilidad de establecer tribunales de evaluación
  competentes en el idioma solicitado.

\item Formato. La redacción se hará en un formato de papel DIN A4. En ellas debe incluirse la
  bibliografía, tablas, gráficos, ilustraciones y anexos o apéndices. Los márgenes serán de 2’5 cm,
  y el interlineado de 1’5, sin espaciado especial entre párrafos. Se empleará un tipo de letra
  Arial 12 y el texto estará justificado. La primera línea de cada párrafo deberá estar indentada.
  La EPSJ proporcionará una plantilla para facilitar la redacción de la memoria. Las páginas
  estarán numeradas en el margen inferior derecho.  
\item \textbf{Guía para la redacción de la memoria.}
  \begin{enumerate}
  \item La portada del TFG deberá contener la siguiente información:
    \begin{itemize}
    \item Escudo de la Universidad
    \item UNIVERSIDAD DE JAÉN (Tamaño de letra: 16‐18)
    \item ESCUELA POLITÉCNICA SUPERIOR DE JAÉN (Tamaño de letra: 16‐18)
    \item Grado en Ingeniería XXXXXXXX (Tamaño de letra: 16‐18)
    \item Trabajo Fin de Grado (Tamaño de letra: 16‐18)
    \item TÍTULO DEL TRABAJO (Tamaño de letra: 30‐36)
    \item  Nombre del alumno (Tamaño de letra: 16‐18)
    \item Lugar y fecha de defensa (Tamaño de letra: 16‐18).
    \end{itemize}
    
    La EPSJ proporcionará una plantilla particularizada para cada título de Grado con el tipo y
    tamaño de fuentes adecuadas conteniendo la anterior información y respetando la normativa
    establecida por la UJA en cuanto a representación y reproducción de los símbolos
    identificativos de la Universidad.
    
  \item En la primera página estarán los mismos datos de la portada con la firma del alumno y con
    el nombre del tutor/a y su firma dando el VºBº a la defensa del TFG.  
  \item Deberá llevar un índice en el que se hará constar los títulos de capítulos y apartados y las
    páginas correspondientes, así como, la bibliografía y los posibles anexos y planos. Tanto en el
    índice como en el trabajo se utilizará un sistema de ordenación decimal (1., 1.1., 1.1.1., etc.)
    que permita visualizar fácilmente la jerarquía de contenidos.
  \item Los capítulos, los apartados y subapartados llevarán numeración arábiga correlativa,
    ordenados por sistema decimal, tal y como se indica en el apartado anterior. Los capítulos
    aparecerán en mayúscula y negrita. Los apartados irán en negrita, con interlineado doble por
    encima e interlineado simple por debajo, y los subapartados en cursiva, dejando un espacio de
    interlineado por encima y otro por debajo.  
  \item Las figuras y tablas deberán citarse en el texto y se intercalarán en el lugar correspondiente
    en el cuerpo del texto después de su cita y lo más próximo posible a ella. Se señalarán con
    numeración arábiga y habrá una numeración diferenciada para las figuras y otra para las
    tablas. La numeración será consecutiva a lo largo de toda la memoria o bien para cada capítulo
    por separado. En este último caso llevarán dos dígitos separados por un punto, el primero de
    los cuales hace referencia al capítulo y el segundo al número de la figura o la tabla (por
    ejemplo, Figura 1.2 o Tabla 3.1). Tanto las figuras como las tablas llevarán su pie
    correspondiente.
    Las figuras y las tablas estarán alineadas horizontalmente con el texto, salvo que por su ancho
    sea aconsejable una orientación vertical, en cuyo caso el pie tendrá la misma alineación que la
    figura o la tabla.
  \item Deben evitarse en la medida de lo posible el uso de pies de página, pero si se requieren debe
    emplearse una numeración arábiga. Los pies de página irán en la parte inferior de esta,  
    separados del texto principal con una línea horizontal y con fuente Arial 10.
  \item Ecuaciones, símbolos y unidades. Las ecuaciones deben estar numeradas consecutivamente
    a lo largo de la memoria. El número de la ecuación (debe servir para citarlo en el texto si se
    requiere) se escribirá entre paréntesis y estará justificado a la derecha. Deben dejarse dos
    líneas antes y después de la ecuación. Por ejemplo:  

    \begin{equation}\label{eq:test}
      \begin{split}
        x &= x_0 - c \frac{X - X_0}{Z - Z_0}\\
        y &= y_0 - c \frac{Y- Y_0}{Z - Z_0}\\
      \end{split}
    \end{equation}

    \begin{equation}
      \begin{split}
        \text{Donde} \quad &c = \text{distancia focal}\\
                           & x, y = \text{coordenadas imagen}\\
                           & X_0, Y_0, Z_0 = \text{coordenadas del centro de proyección}\\
                           & X, Y, Z = \text{coordenadas objeto}\\
      \end{split}
    \end{equation}
    
    Para los símbolos y unidades se empleará el Sistema Internacional (SI). Los símbolos o
    caracteres no usuales deberán estar explicados en una lista de nomenclatura. Se podrán
    emplear otros sistemas de unidades que, por las características de la disciplina en la que se
    encuadre el TFG, sean habituales tanto a nivel nacional como internacional.
  \item En caso que el TFG disponga de bibliografía (citas y referencias), esta se pondrá atendiendo
    a cualquier sistema estandarizado habitualmente empleado en trabajos técnicos y/o
    científicos. Las referencias bibliográficas, ya sean de artículos en revistas, libros, apuntes
    editados, manuales, referencias web, etc., deben estar lo suficientemente completas, de forma
    que cualquier lector potencial pueda encontrar dichas citas en las bases de datos
    bibliográficas.    La EPSJ pondrá a disposición de los estudiantes, a través de su página web,
    diversos sistemas de citas bibliográficas habituales en los trabajos técnicos.
    Hay que indicar que en caso de incorporarse referencias bibliográficas (hay determinados TFG
    que por su naturaleza y modalidad no requieren su incorporación en la memoria), estas
    referencias deben aparecer citadas en  el texto en el lugar que les corresponda y conforme a
    las normas de estilo antes comentadas.
  \end{enumerate}
\end{enumerate}
