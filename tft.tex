% !TEX program = pdflatex
% !TEX encoding = UTF-8 Unicode

% Plantilla, basada en la clase `scrbook` del paquete KOMA-script,  para la elaboración de un TFG o TFM siguiendo las directrices de la Escuela Politécnica Superior de Jaén (https://eps.ujaen.es/sites/centro_epsj/files/uploads/documents/grados/TFG/criteriosYestilo_TFG.pdf)

% Basada en la plantilla para la elaboración de un TFG bajo los requisitos de la comisión del Grado en Matemáticas de la Universidad de Granada (https://github.com/latex-mat-ugr/Plantilla-TFG)
% Francisco Torralbo Torralbo
% Carlos Romero Cruz

\documentclass[print, color]{ujaTFT}

% VERSIÓN ELECTRÓNICA PARA TABLETA
% Cambiando la opción "print" por "tablet" generaremos un pdf adaptado para leerlo en tabletas de 9 pulgadas.

% -------------------------------------------------------------------
% INFORMACIÓN DEL TFG Y EL AUTOR
% -------------------------------------------------------------------

\newcommand{\miTitulo}{Título del trabajo\xspace}
\newcommand{\miNombre}{Nombre apellidos\xspace}
\newcommand{\miGrado}{Nombre del máster}
\newcommand{\miFacultad}{Mi facultad}
\newcommand{\miUniversidad}{Universidad de Jaén}

\newcommand{\miTutor}{Nombre del tutor}
\newcommand{\miTutorDepar}{Nombre del departamento}
\newcommand{\miFecha}{la fecha de defensa\xspace}
\newcommand{\miCurso}{2024-2025\xspace}

\hypersetup{
  pdftitle={\miTitulo},
  pdfauthor={\textcopyright\ \miNombre, \miFacultad, \miUniversidad}
}

\begin{document}

\maketitle

% -------------------------------------------------------------------
% FRONTMATTER
% -------------------------------------------------------------------
\frontmatter % Desactiva la numeración de capítulos y usa numeración romana para las páginas

% !TeX root = ../tft.tex
% !TeX encoding = utf8

%*******************************************************
% Autorización
%*******************************************************

\input{preliminares/dedicatoria}
\input{preliminares/agradecimientos}

\input{preliminares/tablacontenidos}            
\input{preliminares/introduccion}               


% -------------------------------------------------------------------
% MAINMATTER
% -------------------------------------------------------------------
\mainmatter % activa la numeración de capítulos, resetea la numeración de las páginas y usa números arábigos

\part{Primera parte} % Dividir un TFG en partes

% Información relevante para la elaboración del trabajo.
% !TeX root = ../tft.tex
% !TeX encoding = utf8

\chapter{Documentación}\label{ch:primer-capitulo}

\section{Introducción}
Este documento es una plantilla para la elaboración de un Trabajo Fin de Grado y de Máster, en lo que
sigue como Trabajo Final de Título, siguiendo los \href{https://eps.ujaen.es/sites/centro_epsj/files/uploads/documents/grados/TFG/criteriosYestilo_TFG.pdf}{criterios} de la
Escuela Politécnica Superior de Jaén, los cuales son los siguientes:

\begin{enumerate}
\item Aspectos propios del TFT:
  \begin{itemize}
  \item Claridad en el planteamiento general del trabajo.
  \item Adecuación de la estructura y del contenido al tipo de trabajo.
  \item Claridad en el establecimiento de los antecedentes y de los objetivos.
  \item Adecuación de los materiales, métodos, procedimientos y criterios empleados.
  \item Correspondencia entre los objetivos planteados en la propuesta del TFG original
    (aprobada en su momento por la Comisión de TFG)    y la discusión de los resultados
    obtenidos.
  \item Adecuación de las conclusiones a los resultados obtenidos.
  \item Empleo, si procede, de bibliografía adecuada y actualizada.
  \item Específicos para proyectos de ingeniería (además de los anteriores):
    \begin{itemize}
    \item Adaptación a normas de la estructura del proyecto, en su conjunto.
    \item Presencia de todos los documentos necesarios.
    \item Adecuación y adaptación a normas del contenido de cada uno de los apartados documentales.
    \end{itemize}
  \end{itemize}
\item Aspectos formales:
  \begin{itemize}
  \item Calidad en la estructuración del trabajo.
  \item Calidad de la redacción y léxico técnico empleado.
  \item Calidad en la presentación de tablas, figuras y planos.
  \item Calidad en la presentación de la bibliografía (citas y referencias).
  \end{itemize}
\item Aspectos relativos a la defensa del trabajo:
  \begin{itemize}
  \item Claridad en la exposición del trabajo. Se utiliza el vocabulario adecuado en cada
    circunstancia y se hace un uso adecuado del léxico técnico cuando es necesario.
  \item Grado de síntesis y adecuación de la estructura de la exposición.
  \item Calidad del material de apoyo utilizado.
  \item Adecuación al tiempo disponible.
  \item Grado de madurez y de conocimientos demostrado en el debate posterior a la
    exposición.
  \end{itemize}
\end{enumerate}

Para generar el pdf a partir de la plantilla basta compilar el fichero \texttt{tfg.tex}. Es conveniente leer los comentarios contenidos en dicho fichero pues ayudarán a entender mejor como funciona la plantilla. 

La estructura de la plantilla es la siguiente\footnote{Los nombres de las carpetas no se han acentuado para evitar problemas en sistemas con Windows}: 
\begin{itemize}
  \item Carpeta \textbf{preliminares}: contiene los siguientes archivos
    \begin{description}
      \item[\texttt{dedicatoria.tex}] Para la dedicatoria del trabajo (opcional)
      \item[\texttt{agradecimientos.tex}] Para los agradecimientos del trabajo (opcional)
      \item[\texttt{introduccion.tex}] Para la introducción (obligatorio)
      \item[\texttt{tablacontenidos.tex}] Genera de forma automática la tabla de contenidos, el índice de figuras y el índice de tablas. Si bien la tabla de contenidos es conveniente incluirla, el índice de figuras y tablas es opcional. Por defecto está desactivado. Para mostrar dichos índices hay que editar este fichero y quitar el comentario a \verb+\listoffigures+ o \verb+\listoftables+ según queramos uno de los índices o los dos. En este archivo también es posible habilitar la inclusión de un índice de listados de código (si estos han sido incluidos con el paquete \texttt{listings})
  \end{description}
  El resto de archivos de dicha carpeta no es necesario editarlos pues su contenido se generará automáticamente a partir de los metadatos que agreguemos en \texttt{tft.tex}

  \item Carpeta \textbf{capitulos}: contiene los archivos de los capítulos del TFT. Añadir tantos archivos como sean necesarios. Este capítulo es \texttt{capitulo01.tex}.

  \item Carpeta \textbf{apendices}: Para los apéndices (opcional)
  \item Carpeta \textbf{img}: Para incluir los ficheros de imagen que se usarán en el documento.
    
  \item Fichero \texttt{library.bib}: Para incluir las referencias bibliográficas en formato \texttt{bibtex}. Es útil la herramienta \href{https://www.doi2bib.org/}{doi2bib} para generar de forma automática el código bibtex de una referencia a partir de su \textsc{doi}  así como la base de datos bibliográfica \href{https://mathscinet.ams.org}{MathSciNet}. Para que una referencia aparezca en el pdf no basta con incluirla en el fichero \texttt{library.bib}, es necesario además \emph{citarla} en el documento usando el comando \verb+\cite+. Si queremos mostrar todos las referencias incluidas en el fichero \texttt{library.bib} podemos usar \verb+\cite{*}+ aunque esta opción no es la más adecuada. Se aconseja que los elementos de la bibliografía estén citados al menos una vez en el documento (y de esa forma aparecerán de forma automática en la lista de referencias).

  \item Fichero \texttt{glosario.tex}: Para incluir un glosario en el trabajo (opcional). Si no queremos incluir un glosario deberemos borrar el comando \verb+\input{glosario.tex}+ del fichero \texttt{tfg.tex} y posteriormente borrar el fichero \texttt{glosario.tex}

   \item Fichero \texttt{tft.tex}: El documento maestro del TFT que hay que compilar con \LaTeX\ para obtener el pdf. En dicho documento hay que cambiar la \emph{información del título del \textsc{tfg} y el autor así como los tutores}.
\end{itemize}



\section{Elementos del texto}

En esta sección presentaremos diferentes ejemplos de los elementos de texto básico. Conviene consultar el contenido de \texttt{capitulos/capitulo01.tex} para ver cómo se han incluido.

\subsection{Listas}
En \LaTeX\ tenemos disponibles los siguientes tipos de listas:

Listas enumeradas:
\begin{enumerate}
  \item item 1
  \item item 2
  \item item 3
\end{enumerate}

Listas no enumeradas
\begin{itemize}
  \item item 1
  \item item 2
  \item item 3
  \end{itemize}

Listas descriptivas
\begin{description}
  \item[termino1] descripción 1
  \item[termino2] descripción 2
\end{description}
  
\subsection{Tablas y figuras}

En la \autoref{tb:ejemplo-tabla} o la \autoref{fig:logo-uja} podemos ver\ldots

\begin{table}[htpb]
  \centering
  \begin{tabular}{ccc} \toprule
    \multicolumn{2}{c}{Agrupados} \\ \cmidrule(r){1-2}
    cabecera & cabecera & cabecera          \\ \midrule
    elemento & elemento & elemento          \\ 
    elemento & elemento & elemento          \\ 
    elemento & elemento & elemento          \\ \bottomrule
  \end{tabular}
  \caption{Ejemplo de tabla}
  \label{tb:ejemplo-tabla}
\end{table}

\begin{figure}[htpb]
  \centering
  \includegraphics[width=0.3\textwidth]{logo-uja}
  \caption{Logotipo de la Universidad de Jaén}
  \label{fig:logo-uja}
\end{figure}

\section{Entornos matemáticos}\label{sec:entornos-matematicos}

La plantilla tiene definidos varios entornos matemáticos cuyo nombre es el mismo omitiendo los acentos. Así, para incluir una \emph{proposición} usaríamos:

\begin{verbatim}
\begin{proposicion}
texto de la proposición
\end{proposicion} 
\end{verbatim}

Ver el código fuente del archivo \texttt{documentacion.tex} en la carpeta \texttt{capitulos} para el resto de ejemplos.

\begin{teorema}\label{thm:teorema}
Esto es un ejemplo de teorema.
\end{teorema}

\begin{proposicion}
Ejemplo de proposición
\end{proposicion}

\begin{lema}
Ejemplo de lema
\end{lema}

\begin{corolario}
Ejemplo de corolario
\end{corolario}

\begin{definicion}
Ejemplo de definición
\end{definicion}

\begin{observacion}
Ejemplo de observación
\end{observacion}

Adicionalmente está definido el entorno \texttt{teorema*} que permite incluir un teorema sin numeración:

\begin{teorema*}[Fórmula de Gauß-Bonnet]
  Sea $S$ una superficie compacta y $K$ su curvatura de Gauß. Entonces
\begin{equation}
  \int_S K = 2\pi\chi(S)
\end{equation}
donde $\chi(S)$ es la característica de Euler de $S$.
\end{teorema*}

Las fórmulas matemáticas se escriben entre símbolos de dólar \$ si van en línea con el texto o bien usando el entorno%
\footnote{
  También es posible delimitar una ecuación mediante los comandos \texttt{$\backslash$[} y \texttt{$\backslash$]} pero éstas nunca llevarán numeración aunque añadamos una etiqueta y las referenciemos (ver \autoref{sec:referencias}).
} 
\texttt{equation} cuando queremos que se impriman centradas en una línea propia, como el siguiente ejemplo
\begin{equation}\label{eq:identidad-pitagorica}
  \cos^2 x + \sin^2 x = 1.
\end{equation}


Gracias al paquete \texttt{mathtools}, las ecuaciones escritas dentro del entorno \texttt{equation} llevarán numeración de forma automática si son referenciadas  en cualquier parte del documento (por ejemplo la identidad Pitagórica~\eqref{eq:identidad-pitagorica}, ver el código de los dos anteriores ejemplos y la \autoref{sec:referencias} para más información sobre referencias cruzadas en el documento).

\section{Listados de código}

Podemos incluir un archivo externo de código mediante el comando \texttt{lstinputlisting} especificando su nombre completo (incluyendo la extensión) y usando la opción \texttt{inputpath} para indicar la ruta hacia el fichero (siempre referida a la carpeta principal de la plantilla) así como la opción \texttt{language} para indicar el lenguaje de programación en que está escrito (esto permitirá a \LaTeX\ colorear adecuadamente el código). Además, si lo consideramos necesario, podemos indicar las líneas que queremos mostrar (ver el código fuente del \autoref{code:prime}). Consultar todas las opciones posibles en la \href{https://osl.ugr.es/CTAN/macros/latex/contrib/listings/listings.pdf}{documentación del paquete \texttt{listings}}.

\lstinputlisting[inputpath=code, language=R, linerange={11-17}, firstnumber={11}, caption={Extracto código (líneas de 11 a 17) del fichero \texttt{primeR.r}}, label={code:prime}]{primeR.r}

Alternativamente, podemos incluir el código en un entorno \texttt{lstlisting} como el \autoref{code:perceptron}

\begin{lstlisting}[caption={Implementación de un perceptrón}, label={code:perceptron}, language={python}]
def dot(v, w):
    """Producto escalar de v y w, |$\color{comment}v_0 \cdot   w_0 + \cdots + v_n \cdot w_n$|"""
    return sum(v_i * w_i for v_i, w_i in zip(v, w))

def funcion_activacion(x):
    """1 si la entrada es mayor o igual que 1, 0 en otro caso."""
    return 1 if x >= 0 else 0

def perceptron(entrada, pesos):
    """1 si el perceptron se activa, 0 en otro caso"""
    return funcion_activacion(dot(entrada, pesos))
\end{lstlisting}

La opción \texttt{float} al incluir un listado de código permitará a dicho bloque ``flotar'' como si fuese un entorno \texttt{figure} y de esta manera evitaremos que se corte al final de una página.



\section{Referencias a elementos del texto}\label{sec:referencias}

Para las referencias a los elementos del texto (secciones, capítulos, teoremas,\ldots) se procede de la siguiente manera:
\begin{itemize}
  \item Se \emph{marca} el elemento (justo después del mismo si se trata de un capítulo o sección o en el interior del \emph{entorno} en otro caso), mediante el comando \verb+\label{+\emph{etiqueta}\verb+}+, donde \emph{etiqueta} debe ser un identificador único.
  \item Para crear una referencia al elemento en cualquier otra parte del texto se usa el comando \verb+\ref{+\emph{etiqueta}\verb+}+ (únicamente imprime la numeración asociada a dicho elemento, por ejemplo \ref{ch:primer-capitulo} o \ref{thm:teorema}) o bien \verb+\autoref{+\emph{etiqueta}\verb+}+ (imprime la numeración del elemento así como un texto previo indicando su tipo, por ejemplo \autoref{ch:primer-capitulo} o \autoref{thm:teorema})
\end{itemize}




\section{Bibliografía e índice}

Esto es un ejemplo de texto en un capítulo. Incluye varias citas tanto a libros~\cite{Aigner2018}, artículos de investigación~\cite{Euler1985}, recursos online~\cite{EulerWiki} (páginas web), tesis~\cite{CitekeyPhdthesis}, trabajo fin de máster~\cite{CitekeyMastersthesis}, trabajo fin de grado~\cite{CiteKeyBachelorsthesis} así como artículos sin publicar (preprints) \cite{castroinfantes2022conjugate} (en estos últimos usar el campo \texttt{note} para añadir la información relevante). 

Ver el fichero \texttt{library.bib} para las distintas plantillas. Cada nueva referencia debe añadirse en dicho fichero siguiendo el estilo del código \texttt{bibtex} según el tipo de referencia (página web, tesis, trabajo fin de grado o máster, artículo de investigación, libro,\ldots). Alternativamente se puede usar la web \href{https://zbib.org}{https://zbib.org} para generar automáticamente el código \texttt{bibtex}.


\endinput

\chapter[Consideraciones elaboración TFG]{Consideraciones generales para la elaboración de un Trabajo Fin de Título}

\section{Normativa de la Universidad de Jaén y la Escuela Politécnica Superior de Jaén}

El \textsc{tft} lo rigen dos normativas: 
\begin{itemize}
  \item una a nivel general de la UJA (\href{https://www.ujaen.es/gobierno/secgen/sites/gobierno_secgen/files/uploads/CG25_ANEXO04_Normativa_Trabajos\%20_TFG_TFM_otros_TFT_UJA\%20CG\%2025\%20de\%205\%20junio\%202017.pdf}{Normativa marco UJA aprobada en Consejo de Gobierno (2017)}\footnote{\url{https://www.ujaen.es/gobierno/secgen/sites/gobierno_secgen/files/uploads/CG25_ANEXO04_Normativa_Trabajos\%20_TFG_TFM_otros_TFT_UJA\%20CG\%2025\%20de\%205\%20junio\%202017.pdf}}) y 
  \item otra complementaria a nivel de la Escuela Politécnica Superior de Jaén  (\href{https://eps.ujaen.es/sites/centro_epsj/files/uploads/documents/normativa/Normativa_TFG_TFM_EPSJ_aprobada\%20Junta\%20de\%20Escuela\%2013\%20sept\%202017.pdf}{Normativa EPSJ aprobada en Junta de Escuela (2017)}\footnote{\url{https://eps.ujaen.es/sites/centro_epsj/files/uploads/documents/normativa/Normativa_TFG_TFM_EPSJ_aprobada\%20Junta\%20de\%20Escuela\%2013\%20sept\%202017.pdf}}). 
\end{itemize}

Toda la información anterior puede encontrarse en la \href{https://eps.ujaen.es/principal/trabajo-fin-de-grado-master}{web de la Escuela Politécnia Superior de Jaén}\footnote{\url{https://eps.ujaen.es/principal/trabajo-fin-de-grado-master}}

\section{Normas de estilo}
Las normas de estilo para la realización del TFT se adaptarán en función de la modalidad de
TFT elegida.

\begin{itemize}
\item Los Proyectos de Ingeniería deberán presentar una estructura y contenido adaptados
a la norma UNE 157001 “Criterios generales para la elaboración de proyectos” u otra
derivada de ésta, en función de la naturaleza del proyecto.
Estos aspectos podrán detallarse bien en la guía docente y en la propuesta del TFG que
presente el departamento y apruebe la Comisión de TFG de la EPSJ. Por ejemplo, para
el caso de los sistemas de información informatizados (sistemas informáticos, sistemas
de información geográfica, etc.) está la norma derivada de la anterior UNE 157801
“Criterios generales para la elaboración de proyectos de sistemas de información”).
En general, el TFG deberá incluir los siguientes documentos básicos: Índice General,
Memoria, Anexos, Planos, Pliego de Condiciones , Estado de Mediciones, Presupuesto
y, cuando proceda, Estudios con Entidad Propia, presentados en el orden indicado. Los
planos, en caso de ser necesarios (los de sistemas de información, p.e., pueden no
llevar planos) se adecuarán a las normas UNE de Dibujo Técnico.  
\item Los Estudios Técnicos, debido a su singularidad, podrán tener una estructura variable
según su naturaleza. Como mínimo, deberían constar de los apartados siguientes:
Introducción, Objetivos, Discusión, Conclusiones, Planos y Anexos (si proceden) y
Bibliografía.
\item Los Trabajos teóricos o experimentales se estructurarán, en la medida de lo posible,
en los siguientes apartados: Introducción, Antecedentes, Objetivos, Material y
Métodos, Resultados y Discusión, Conclusiones, Planos y Anexos (si proceden) y
Bibliografía.
\end{itemize}

\begin{enumerate}
\item Idioma. La memoria del TFG se podrá elaborar en un idioma distinto al castellano, bajo
  petición del estudiante y del tutor/a a la Comisión de TFG de la Escuela, siempre que el idioma
  elegido por el alumno se encuentre entre los que se han utilizado en la impartición del grado.
  En este caso, se deberá proporcionar al menos un resumen con la introducción y las
  conclusiones del TFG en castellano. La Comisión de TFG de la Escuela valorará las peticiones, y
  accederá a ellas siempre que se tenga la posibilidad de establecer tribunales de evaluación
  competentes en el idioma solicitado.

\item Formato. La redacción se hará en un formato de papel DIN A4. En ellas debe incluirse la
  bibliografía, tablas, gráficos, ilustraciones y anexos o apéndices. Los márgenes serán de 2’5 cm,
  y el interlineado de 1’5, sin espaciado especial entre párrafos. Se empleará un tipo de letra
  Arial 12 y el texto estará justificado. La primera línea de cada párrafo deberá estar indentada.
  La EPSJ proporcionará una plantilla para facilitar la redacción de la memoria. Las páginas
  estarán numeradas en el margen inferior derecho.  
\item \textbf{Guía para la redacción de la memoria.}
  \begin{enumerate}
  \item La portada del TFG deberá contener la siguiente información:
    \begin{itemize}
    \item Escudo de la Universidad
    \item UNIVERSIDAD DE JAÉN (Tamaño de letra: 16‐18)
    \item ESCUELA POLITÉCNICA SUPERIOR DE JAÉN (Tamaño de letra: 16‐18)
    \item Grado en Ingeniería XXXXXXXX (Tamaño de letra: 16‐18)
    \item Trabajo Fin de Grado (Tamaño de letra: 16‐18)
    \item TÍTULO DEL TRABAJO (Tamaño de letra: 30‐36)
    \item  Nombre del alumno (Tamaño de letra: 16‐18)
    \item Lugar y fecha de defensa (Tamaño de letra: 16‐18).
    \end{itemize}
    
    La EPSJ proporcionará una plantilla particularizada para cada título de Grado con el tipo y
    tamaño de fuentes adecuadas conteniendo la anterior información y respetando la normativa
    establecida por la UJA en cuanto a representación y reproducción de los símbolos
    identificativos de la Universidad.
    
  \item En la primera página estarán los mismos datos de la portada con la firma del alumno y con
    el nombre del tutor/a y su firma dando el VºBº a la defensa del TFG.  
  \item Deberá llevar un índice en el que se hará constar los títulos de capítulos y apartados y las
    páginas correspondientes, así como, la bibliografía y los posibles anexos y planos. Tanto en el
    índice como en el trabajo se utilizará un sistema de ordenación decimal (1., 1.1., 1.1.1., etc.)
    que permita visualizar fácilmente la jerarquía de contenidos.
  \item Los capítulos, los apartados y subapartados llevarán numeración arábiga correlativa,
    ordenados por sistema decimal, tal y como se indica en el apartado anterior. Los capítulos
    aparecerán en mayúscula y negrita. Los apartados irán en negrita, con interlineado doble por
    encima e interlineado simple por debajo, y los subapartados en cursiva, dejando un espacio de
    interlineado por encima y otro por debajo.  
  \item Las figuras y tablas deberán citarse en el texto y se intercalarán en el lugar correspondiente
    en el cuerpo del texto después de su cita y lo más próximo posible a ella. Se señalarán con
    numeración arábiga y habrá una numeración diferenciada para las figuras y otra para las
    tablas. La numeración será consecutiva a lo largo de toda la memoria o bien para cada capítulo
    por separado. En este último caso llevarán dos dígitos separados por un punto, el primero de
    los cuales hace referencia al capítulo y el segundo al número de la figura o la tabla (por
    ejemplo, Figura 1.2 o Tabla 3.1). Tanto las figuras como las tablas llevarán su pie
    correspondiente.
    Las figuras y las tablas estarán alineadas horizontalmente con el texto, salvo que por su ancho
    sea aconsejable una orientación vertical, en cuyo caso el pie tendrá la misma alineación que la
    figura o la tabla.
  \item Deben evitarse en la medida de lo posible el uso de pies de página, pero si se requieren debe
    emplearse una numeración arábiga. Los pies de página irán en la parte inferior de esta,  
    separados del texto principal con una línea horizontal y con fuente Arial 10.
  \item Ecuaciones, símbolos y unidades. Las ecuaciones deben estar numeradas consecutivamente
    a lo largo de la memoria. El número de la ecuación (debe servir para citarlo en el texto si se
    requiere) se escribirá entre paréntesis y estará justificado a la derecha. Deben dejarse dos
    líneas antes y después de la ecuación. Por ejemplo:  

    \begin{equation}\label{eq:test}
      \begin{split}
        x &= x_0 - c \frac{X - X_0}{Z - Z_0}\\
        y &= y_0 - c \frac{Y- Y_0}{Z - Z_0}\\
      \end{split}
    \end{equation}

    \begin{equation}
      \begin{split}
        \text{Donde} \quad &c = \text{distancia focal}\\
                           & x, y = \text{coordenadas imagen}\\
                           & X_0, Y_0, Z_0 = \text{coordenadas del centro de proyección}\\
                           & X, Y, Z = \text{coordenadas objeto}\\
      \end{split}
    \end{equation}
    
    Para los símbolos y unidades se empleará el Sistema Internacional (SI). Los símbolos o
    caracteres no usuales deberán estar explicados en una lista de nomenclatura. Se podrán
    emplear otros sistemas de unidades que, por las características de la disciplina en la que se
    encuadre el TFG, sean habituales tanto a nivel nacional como internacional.
  \item En caso que el TFG disponga de bibliografía (citas y referencias), esta se pondrá atendiendo
    a cualquier sistema estandarizado habitualmente empleado en trabajos técnicos y/o
    científicos. Las referencias bibliográficas, ya sean de artículos en revistas, libros, apuntes
    editados, manuales, referencias web, etc., deben estar lo suficientemente completas, de forma
    que cualquier lector potencial pueda encontrar dichas citas en las bases de datos
    bibliográficas.    La EPSJ pondrá a disposición de los estudiantes, a través de su página web,
    diversos sistemas de citas bibliográficas habituales en los trabajos técnicos.
    Hay que indicar que en caso de incorporarse referencias bibliográficas (hay determinados TFG
    que por su naturaleza y modalidad no requieren su incorporación en la memoria), estas
    referencias deben aparecer citadas en  el texto en el lugar que les corresponda y conforme a
    las normas de estilo antes comentadas.
  \end{enumerate}
\end{enumerate}


% Añadir tantos capítulos como sea necesario

\cleardoublepage\part{Segunda parte}
\input{capitulos/capitulo-ejemplo}

% -------------------------------------------------------------------
% APPENDIX: Opcional
% -------------------------------------------------------------------

\appendix % Reinicia la numeración de los capítulos y usa letras para numerarlos
\pdfbookmark[-1]{Apéndices}{appendix} % Alternativamente podemos agrupar los apéndices con un nuevo \part{Apéndices}

\input{apendices/apendice-ejemplo}
% Añadir tantos apéndices como sea necesario 

% -------------------------------------------------------------------
% GLOSARIO
% -------------------------------------------------------------------

\input{glosario} 

% -------------------------------------------------------------------
% BACKMATTER
% -------------------------------------------------------------------

\backmatter % Desactiva la numeración de los capítulos
\pdfbookmark[-1]{Referencias}{BM-Referencias}

% BIBLIOGRAFÍA
%-------------------------------------------------------------------

\bibliographystyle{alpha-es} 
\begin{small} 
  \bibliography{library.bib}
\end{small}


\end{document}
